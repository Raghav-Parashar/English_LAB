\documentclass[a4paper,11pt]{article}
\usepackage[utf8]{inputenc}
\usepackage[margin=0.8cm]{geometry} % Reduced margins to fit content
\usepackage{xcolor}
\usepackage{tcolorbox}
\usepackage{enumitem}
\usepackage{times}
\definecolor{navyblue}{RGB}{0, 51, 102}
\definecolor{gold}{RGB}{204, 153, 0}
\definecolor{lightgray}{RGB}{230, 230, 230}

\title{}
\author{}
\date{}

\begin{document}

\begin{center}
    {\LARGE\bfseries\color{navyblue} Parts of Speech in English Communication} \\
    \vspace{0.1cm}
    {\small\itshape A Fundamental Framework for Effective Language Use} \\
    \vspace{0.1cm}
    \hrule height 1pt
    \vspace{0.1cm}
\end{center}

\begin{flushright}
    \small \textbf{Prepared by:} Raghav Parashar \\
    \small \textbf{Roll No:} 24WU0102104
    \vspace{0.2cm}
\end{flushright}

{\footnotesize \textbf{Introduction:} The parts of speech form the foundation of English grammar, enabling precise and effective communication. Understanding their roles enhances clarity, coherence, and expressiveness in both written and spoken language. This chart explores the eight parts of speech, their functions, and their significance in professional and academic contexts.}

\vspace{0.2cm}

\begin{tcolorbox}[colback=lightgray, colframe=navyblue, sharp corners, boxrule=0.5mm, title=\textbf{1. Noun: The Core of Identification}, boxsep=2pt, left=2pt, right=2pt, top=2pt, bottom=2pt]
    \footnotesize \textbf{Definition:} A noun identifies a person, place, object, or concept. \\
    \textbf{Types:} Common (e.g., city), Proper (e.g., Paris), Abstract (e.g., freedom), Collective (e.g., team). \\
    \textbf{Examples:} The \textit{manager} presented a \textit{strategy} in \textit{New York}. \\
    \textbf{Role:} Nouns anchor sentences, ensuring clarity in identifying subjects or objects.
\end{tcolorbox}

\begin{tcolorbox}[colback=lightgray, colframe=navyblue, sharp corners, boxrule=0.5mm, title=\textbf{2. Pronoun: Enhancing Efficiency}, boxsep=2pt, left=2pt, right=2pt, top=2pt, bottom=2pt]
    \footnotesize \textbf{Definition:} A pronoun substitutes for a noun to avoid redundancy. \\
    \textbf{Types:} Personal (e.g., I, you), Possessive (e.g., mine), Reflexive (e.g., myself). \\
    \textbf{Examples:} \textit{She} completed \textit{her} report, which \textit{itself} was challenging. \\
    \textbf{Role:} Pronouns streamline sentences, making communication concise.
\end{tcolorbox}

\begin{tcolorbox}[colback=lightgray, colframe=navyblue, sharp corners, boxrule=0.5mm, title=\textbf{3. Verb: The Engine of Action}, boxsep=2pt, left=2pt, right=2pt, top=2pt, bottom=2pt]
    \footnotesize \textbf{Definition:} A verb expresses an action, occurrence, or state of being. \\
    \textbf{Types:} Action (e.g., write), Linking (e.g., seem), Auxiliary (e.g., have). \\
    \textbf{Examples:} The team \textit{collaborated} and \textit{was} supported by experts. \\
    \textbf{Role:} Verbs drive the narrative, conveying what happens or exists.
\end{tcolorbox}

\begin{tcolorbox}[colback=lightgray, colframe=navyblue, sharp corners, boxrule=0.5mm, title=\textbf{4. Adjective: The Descriptor}, boxsep=2pt, left=2pt, right=2pt, top=2pt, bottom=2pt]
    \footnotesize \textbf{Definition:} An adjective modifies a noun, providing detail or extent. \\
    \textbf{Types:} Descriptive (e.g., tall), Quantitative (e.g., three), Demonstrative (e.g., that). \\
    \textbf{Examples:} The \textit{innovative} project attracted \textit{several} investors. \\
    \textbf{Role:} Adjectives enrich descriptions, making language vivid and precise.
\end{tcolorbox}

\begin{tcolorbox}[colback=lightgray, colframe=navyblue, sharp corners, boxrule=0.5mm, title=\textbf{5. Adverb: The Modifier of Action}, boxsep=2pt, left=2pt, right=2pt, top=2pt, bottom=2pt]
    \footnotesize \textbf{Definition:} An adverb modifies a verb, adjective, or adverb, indicating manner, time, or degree. \\
    \textbf{Types:} Manner (e.g., carefully), Time (e.g., yesterday), Degree (e.g., very). \\
    \textbf{Examples:} She spoke \textit{eloquently} \textit{yesterday} and was \textit{extremely} persuasive. \\
    \textbf{Role:} Adverbs add nuance, specifying how, when, or to what extent.
\end{tcolorbox}

\begin{tcolorbox}[colback=lightgray, colframe=navyblue, sharp corners, boxrule=0.5mm, title=\textbf{6. Preposition: The Connector}, boxsep=2pt, left=2pt, right=2pt, top=2pt, bottom=2pt]
    \footnotesize \textbf{Definition:} A preposition indicates relationships of time, place, or direction. \\
    \textbf{Types:} Time (e.g., during), Place (e.g., above), Direction (e.g., toward). \\
    \textbf{Examples:} The meeting \textit{during} the conference was held \textit{above} the city. \\
    \textbf{Role:} Prepositions provide context, linking ideas spatially or temporally.
\end{tcolorbox}

\begin{tcolorbox}[colback=lightgray, colframe=navyblue, sharp corners, boxrule=0.5mm, title=\textbf{7. Conjunction: The Linker}, boxsep=2pt, left=2pt, right=2pt, top=2pt, bottom=2pt]
    \footnotesize \textbf{Definition:} A conjunction connects words, phrases, or clauses. \\
    \textbf{Types:} Coordinating (e.g., and), Subordinating (e.g., because), Correlative (e.g., either…or). \\
    \textbf{Examples:} The report was detailed \textit{and} insightful \textit{because} it was researched. \\
    \textbf{Role:} Conjunctions ensure logical flow, combining ideas effectively.
\end{tcolorbox}

\begin{tcolorbox}[colback=lightgray, colframe=navyblue, sharp corners, boxrule=0.5mm, title=\textbf{8. Interjection: The Expression of Emotion}, boxsep=2pt, left=2pt, right=2pt, top=2pt, bottom=2pt]
    \footnotesize \textbf{Definition:} An interjection conveys emotion or exclamation, often standing alone. \\
    \textbf{Types:} Expressive (e.g., alas), Conversational (e.g., well). \\
    \textbf{Examples:} \textit{Alas}, the deadline passed; \textit{well}, let’s regroup. \\
    \textbf{Role:} Interjections add emotional depth, often used in rhetorical contexts.
\end{tcolorbox}

\vspace{0.2cm}
\begin{center}
    \hrule height 1pt
    \vspace{0.1cm}
    {\footnotesize\color{navyblue} \textbf{Conclusion:} Proficiency in the parts of speech empowers communicators to construct sentences with precision, coherence, and impact, essential for academic and professional success.}
\end{center}

\end{document}